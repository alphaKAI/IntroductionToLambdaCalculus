\chapter{はじめに}{
	\section{本書について}{
		本書はalphaKAIがラムダ計算を学習するにあたり、自らの備忘録としてとったノートである(という予定)。 \\
		従っていかなる誤字や脱字等の誤謬を犯しているかは分からない。 \\
		それゆえ、もしも誤記/誤解など修正/訂正が必要な箇所が存在した場合、下記に示す筆者の連絡先まで報告していただけるとうれしい。 \\
		ラムダ計算には「型なしラムダ計算」と「型付きラムダ計算」の2つの計算体系が存在するが、本書では「型なしラムダ計算」について記述する。
		また、本書と平行して型理論についても学習し、本書のようにまとめるつもりである。「型付きラムダ計算」については、型理論について学習後に学習する予定である。
		よって、そちらについては型理論の方のノートを参照してもらいたい(いつ頃「型付きラムダ計算」についての記述を開始できるかは未定であるが)。\\
		\\
		さて、前述したとおり本書は私の備忘録にすぎないが、ラムダ計算に興味・関心がある読者の学習一助になれば幸いである。\\
		さらにはラムダ計算を耳にしたことのない読者がラムダ計算に興味を持つきっかけとなってくれれば筆者にとっては至上の喜びである。

		\subsection{筆者の連絡先}{
			\begin{description}
				\item[Twitter] - @alpha\_kai\_NET
				\item[Mail] - alpha.kai.net@alpha-kai-net.info
			\end{description}
		}
	}
	
	\section{対象とする読者}{
		筆者はラムダ計算に興味を持っている素人に過ぎず、ラムダ計算や計算機科学についての専門家ではない。それゆえ、この文章が学習にとって適しているかどうかも判断出来かねる。
		従って、本書がどのような読者にとって理解可能なものであるかもわからない。
		また、先程述べたようにこれはあくまでも筆者のノートであり、備忘録にすぎない。
		対象とする読者に求める能力/知識については分からないが、少なくとも、何らかのプログラミング言語に関する知識やPCの動作原理に興味とラムダ計算を学習する意欲を持っている方であればなんとかなるだろう。
		また、本質的ではないので、本書中のプログラムのコードなどについての解説はしばしば省略するので、それについては読者が各自で理解していただきたい。
	}
	
	\section{参考資料}{
		本書を執筆するにあたり、下に示す資料を参考にした。実際に参照した資料はまだまだ多いのだが、そのすべてを把握することはできていないため、これらは一部である。
		
		\begin{description}
			\item[Wikipedia(日本語版) - ラムダ計算] : \url{https://ja.wikipedia.org/wiki/ラムダ計算}
			\item[Wikipedia(English) - Lambda calculus] \url{https://en.wikipedia.org/wiki/Lambda_calculus}
		\end{description}

	}
}